% !TEX TS-program = xelatex
% !TEX encoding = UTF-8
%
%
\documentclass[a4paper,12pt]{article}
\usepackage{ctex}
\usepackage{geometry,anysize,changepage,calc,hyperref,cite,fancyhdr,setspace}
\usepackage{amsmath,amssymb,amsthm,listings,xcolor}
\usepackage{graphicx,wrapfig,multirow,diagbox,caption, subcaption,verbatim}
\usepackage{caption,algorithm,algpseudocode}
\usepackage{url,natbib}
\bibliographystyle{abbrvnat}
\setcitestyle{open={(},authoryear,close={)}}
\renewcommand{\algorithmicrequire}{\textbf{输入}}
\renewcommand{\algorithmicensure}{\textbf{输出}}
\floatname{algorithm}{子程序}
\listfiles
%initialset.tex
%
\hypersetup{bookmarksopen=true,colorlinks=true,linkcolor=red,anchorcolor=blue,citecolor=green,linktocpage=true}
\captionsetup{font=footnotesize}
\captionsetup[sub]{font+=footnotesize}
\linespread{1.5}
\marginsize{2.13cm}{2.14cm}{2.4cm}{2.4cm}
\pagestyle{fancy}
\lstset{xleftmargin=1.5em,xrightmargin=1.5em,frame=lines}%,numbers=left,numberstyle=\tiny}
\chead{}\rhead{\thepage}\lhead{\leftmark}\cfoot{}\rfoot{}\lfoot{}

\newcommand{\vp}{\varphi}
\newcommand{\al}{\alpha}
\newcommand{\be}{\beta}
\newcommand{\ti}{\tilde}
\newcommand{\ve}{\varepsilon}
\newcommand{\de}{\delta}
\newcommand{\na}{\nabla}
\newcommand{\pd}{\partial}
\newcommand{\ud}{\mathrm{d}}
\newcommand{\mr}{\mathrm{R}}
\newcommand{\ms}{\mathbb{S}}
\newcommand{\mz}{\mathbb{Z}}
\newcommand{\mn}{\mathbb{N}}
\newcommand{\mc}{\mathbb{C}}
\newcommand{\one}{\textbf{1}}
\newcommand{\prox}{\textbf{prox}}
\DeclareMathOperator*{\argmax}{argmax}
\DeclareMathOperator*{\argmin}{argmin}
%\DeclareMathOperator*{\logg}{log}
%\DeclareMathOperator*{\det}{det}
\DeclareMathOperator*{\tr}{tr}
\DeclareMathOperator*{\st}{s.t.}

\theoremstyle{nonumberplain}
%\theoremheaderfont{\itshape}
%\theorembodyfont{\upshape}
%\theoremseparator{.}
%\theoremsymbol{\ensuremath{\square}
\newtheorem{definition}{定义}
\newtheorem{theorem}{定理}
\newtheorem{lemma}{引理}
%\newtheorem{proof}{证明}

%\renewcommand{\figurename}{{\zihao{5}图}}
%\renewcommand{\tablename}{{\zihao{5}表}}
%\makeatletter %下面两个命令用于创建非浮动体图表的标题
%  \newcommand\figcaption{\def\@captype{figure}\caption} 
%  \newcommand\tabcaption{\def\@captype{table}\caption} 
%\makeatother
%%\renewcommand{\abstractname}{摘\ 要}
%\renewcommand{\contentsname}{目\ 录}
%\renewcommand{\refname}{参考文献}

%\renewcommand{\thetable}{\arabic{section}.\arabic{table}}
%\renewcommand{\thefigure}{\arabic{section}.\arabic{figure}}
%\renewcommand{\theequation}{\arabic{section}.\arabic{equation}}

\makeatletter\@addtoreset{table}{section}\@addtoreset{figure}{section}\@addtoreset{equation}{section}\makeatother




\linespread{1.5}
\author{龙子超}
\title{{\heiti {\zihao{3} 数学分析II-习题课}}}
\date{}
\begin{document}
\maketitle
%===================正文====================
%\begin{abstract}
%\begin{spacing}{1.0}
%
%\end{spacing}
%\end{abstract}

%\tableofcontents
%\newpage

本习题答案集所给出的解答尽可能从教材出发. 课程教材为《数学分析》I-III, 伍胜健编著,
北京大学出版社.
\section*{2018-Mar-14}
\noindent II Chap7 46. 求球体$x^2+y^2+z^2\leq1$和柱体$(x-\frac{1}{2})^2+y^2\leq \frac{1}{4}$公共部分的体积.
\begin{proof}[解1]
  对每个给定的$x_0$, 我们可以求$x=x_0$这个平面与题设区域的截面面积. 截面区域为:
  \[P_0:\left\{\begin{array}{l}
    y^2+z^2\leq1-x_0^2\\
    y^2\leq x_0-x_0^2
  \end{array}
  \right.
    \]
    $P_0$的面积$S_0$为
    \[S_0=\int_{-\sqrt{x_0-x_0^2}}^{\sqrt{x_0-x_0^2}}\sqrt{1-x_0^2-y^2}\ud y, \]
    做积分代换$y=\sqrt{1-x_0^2}\sin\theta,b=\sqrt{\frac{x_0-x_0^2}{1-x_0^2}}=\sqrt{\frac{x_0}{1+x_0}}$,
    \[S_0=2(1-x_0^2)\arcsin\sqrt{\frac{x_0}{1+x_0}}+2(1-x_0)\sqrt{x_0}\]
    因此, 题设区域的体积为
    \begin{eqnarray*}
      V&=&\int_0^1 2(1-x^2)\arcsin\sqrt{\frac{x}{1+x}}+2(1-x)\sqrt{x}, [x=t^2]\\
      &=&\int_0^12\ud(x-\frac{1}{3}x^3)\arctan\sqrt{x}+\int_0^14(1-t^2)t^2\ud t, [\ud(\arctan\sqrt{x})=\frac{1}{2(1+x)\sqrt{x}}]\\
      &=&2(x-\frac{1}{3}x^3)\arctan\sqrt{x}|_0^1+\frac{8}{15}-\int_0^1(x-\frac{1}{3}x^3)\frac{1}{(1+x)\sqrt{x}}\ud x,
    \end{eqnarray*}
\end{proof}




%\begin{algorithm}
%  \caption*{ADMM求解$\min_{X\succeq0}|X|_1,\st |SX-I|\leq\sigma$}
%  %  \caption{ADMM求解$\min_{X\succeq0}|X|_1,\st |SX-I|\leq\sigma$}\label{alg:ADMM}
%  \begin{algorithmic}
%    \Require $S,\sigma,\rho$
%    \Ensure $X,|X|_1$
%    \State \textbf{Repeat while not convergence}
%    \begin{enumerate}
%      \item  根据式(\ref{eq:updateW3_2}
%      $\sim$\ref{eq:updateZ3_2}),依次求$W^+,X^+,Y^+,Z^+$;
%      \item  根据式(\ref{eq:updateU3_2}),依次求${U^1}^+,{U^2}^+,{U^3}^+$;
%    \end{enumerate}
%    \Return $X,|X|_1$
%  \end{algorithmic}
%\end{algorithm}

%\begin{lstlisting}[language=Matlab,keywordstyle=\color{blue},commentstyle=\color{red!80!green!80!blue!80}, rulesepcolor=\color{red!50!green!50!blue!50},tabsize=4]
%
%\end{lstlisting}

%\begin{figure}[htbp!]
%\centering
%\makebox[\textwidth][c] {
%\includegraphics[width=0.9\paperwidth]{}
%}
%\caption{}\label{}
%\end{figure}
%
%\begin{figure}[htbp!]
%   \centering
%   \begin{subfigure}{.48\textwidth}
%	\centering
%	\includegraphics[width = \textwidth]{}
%	\caption{} \label{}
%   \end{subfigure} 
%   \begin{subfigure}{.48\textwidth}
%	\centering
%	\includegraphics[width = \textwidth]{}
%	\caption{} \label{}
%   \end{subfigure}
%   \caption{} \label{}
%\end{figure}

%\begin{table}[htbp!]
%\centering
%\caption{} 
%\begin{subfigure}{0.49\textwidth}
%   \centering
%   \caption{}
%   \begin{tabular}{}
%
%   \end{tabular}
%\end{subfigure}
%\begin{subfigure}{.49\textwidth}
%   \centering
%   \caption{} 
%   \begin{tabular}{}
%
%   \end{tabular}    
%\end{subfigure}
%\end{table}

%\begin{centering}
%\section*{致谢}\addcontentsline{toc}{section}{致谢}
%\end{centering}

%\newpage
%\begin{thebibliography}{}
%\addcontentsline{toc}{section}{参考文献}
%\bibitem{cite_ADMM}Boyd S, Parikh N, Chu E, et al. Distributed Optimization and Statistical Learning via the Alternating Direction Method of Multipliers[J]. Foundations \& Trends® in Machine Learning, 2011, 3(1):1-122.
%\bibitem{cite_FISTA}Beck A, Teboulle M. A Fast Iterative Shrinkage-Thresholding Algorithm for Linear Inverse Problems[J]. Siam Journal on Imaging Sciences, 2009, 2(1):183-202.
%\bibitem{cite_convexoptimization}Boyd S, Vandenberghe L. Convex Optimization[M]. Cambridge University Press, 2004.
%%\bibitem{cite_quasicrystalsgreen}刘有延,傅秀军.\emph{准晶体}[M]. 上海科技教育出版社,1999.
%\end{thebibliography}

\bibliography{ref}
\end{document}



